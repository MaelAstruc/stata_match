\documentclass[11pt]{beamer}

\usepackage[english]{babel}
\usepackage[utf8]{inputenc}
\usepackage[dvipsnames]{xcolor}
\usepackage{graphicx}
\usepackage{tikz}

\let\newsim\sim
\renewcommand{\newsim}{\mathrel{\mathpalette\bueckirel\sim}}
\newcommand{\bueckirel}[2]{%
  \raisebox{\depth}{\scalebox{0.808}{$#1#2$}}%
}

\mode<presentation>
{
  \usetheme{Boadilla}
  \usecolortheme{default}
  \usefonttheme{default}
  \setbeamertemplate{navigation symbols}{}
  \setbeamertemplate{caption}[numbered]
} 

\input{stata-lstlisting}

\title{Pattern matching in Stata}
\subtitle{Chasing the devil in the details}

\author[Mael Astruc--Le Souder]{Mael Astruc--Le Souder}
\institute[BSE]{Bordeaux School of Economics, France}

\date{2024 Stata UK Conference}

\begin{document}

\maketitle

\begin{frame}[fragile]{Motivation}
In research, most of the time is spent preparing the data. \\~

In Stata, most of this can be summarized with two statements: \\~
\begin{lstlisting}[language=Stata]
generate x = ... if ...
replace  x = ... if ...
\end{lstlisting}
\end{frame}

\begin{frame}[fragile]{Find the bug}
Example: You have a variable $like$ with the answers for the question: \\~

\qquad \textit{Do you like Stata ?} \\

\qquad \quad \qquad 1 \qquad 2 \qquad 3 \qquad 4 \qquad 5 \\
\qquad (Absolutely not) \qquad \qquad \qquad (Totally) \\~

You want to summarize it with a new variable $opinion$ \\
\begin{lstlisting}[language=Stata]
generate opinion = "negative" if like <= 2
replace  opinion = "neutral"  if like == 3
replace  opinion = "positive" if like >= 4
\end{lstlisting}
\end{frame}

\begin{frame}[fragile]{Proposition}
The $pmatch$ command
\begin{itemize}
    \item A syntax similar to $switch$ / $match$ statements in other languages
    \item Exhaustiveness and usefulness checks \\~
\end{itemize}

\begin{lstlisting}[language=Stata]
pmatch opinion, variables(like) body(
    1<@$\newsim$@>2 => "negative",
    3   => "neutral",
    4<@$\newsim$@>5 => "positive"
)
    <@\textcolor{red}{Warning : Missing cases}@>
        <@\textcolor{red}{.}@>
\end{lstlisting}
\end{frame}

\begin{frame}[fragile]{Syntax overview}
\begin{lstlisting}[language=Stata]
pmatch varname, Variables(varlist) Body(
    [pattern => exp,]
    [pattern => exp,]
    ...
) [nocheck]
\end{lstlisting}
\end{frame}

\begin{frame}[fragile]{Syntax comparison}
\begin{lstlisting}[language=Stata]
generate varname = exp1 if conditions1 on varlist
replace  varname = exp2 if conditions2 on varlist
\end{lstlisting}

\begin{lstlisting}[language=Stata]
pmatch varname, variables(varlist) body(
    [pattern1 => exp1,]
    [pattern2 => exp2,]
)
\end{lstlisting}
\end{frame}

\begin{frame}[fragile]{Syntax: varname}
\textcolor{LimeGreen}{varname}: the name of the variable you want to modify. \\~

\begin{lstlisting}[language=Stata]
generate <@\textcolor{LimeGreen}{varname}@> = exp1 if conditions1 on varlist
replace  <@\textcolor{LimeGreen}{varname}@> = exp2 if conditions2 on varlist
\end{lstlisting}

\begin{lstlisting}[language=Stata]
pmatch <@\textcolor{LimeGreen}{varname}@>, variables(varlist) body(
    [pattern1 => exp1,]
    [pattern2 => exp2,]
)
\end{lstlisting}
\end{frame}

\begin{frame}[fragile]{Syntax: expressions}
\textcolor{RoyalBlue}{exp}: the new values you want. \\~

\begin{lstlisting}[language=Stata]
generate varname = <@\textcolor{RoyalBlue}{exp1}@> if conditions1 on varlist
replace  varname = <@\textcolor{Purple}{exp2}@> if conditions2 on varlist
\end{lstlisting}

\begin{lstlisting}[language=Stata]
pmatch varname, variables(varlist) body(
    [pattern1 => <@\textcolor{RoyalBlue}{exp1}@>,]
    [pattern2 => <@\textcolor{Purple}{exp2}@>,]
)
\end{lstlisting}
\end{frame}

\begin{frame}[fragile]{Syntax: conditions}
\textcolor{RoyalBlue}{conditions}/\textcolor{RoyalBlue}{patterns}: the conditions for the replacements. \\~

\begin{lstlisting}[language=Stata]
generate varname = exp1 if <@\textcolor{RoyalBlue}{conditions1 on varlist}@>
replace  varname = exp2 if <@\textcolor{Purple}{conditions2 on varlist}@>
\end{lstlisting}

\begin{lstlisting}[language=Stata]
pmatch varname, variables(varlist) body(
    [<@\textcolor{RoyalBlue}{pattern1}@> => exp1,]
    [<@\textcolor{Purple}{pattern2}@> => exp2,]
)
\end{lstlisting}
\end{frame}

\begin{frame}[fragile]{Syntax: varlist}
\textcolor{OrangeRed}{varlist} : the variables that determine the replacement. \\~

\begin{lstlisting}[language=Stata]
generate varname = exp1 if conditions1 on <@\textcolor{OrangeRed}{varlist}@>
replace  varname = exp2 if conditions2 on <@\textcolor{OrangeRed}{varlist}@>
\end{lstlisting}

\begin{lstlisting}[language=Stata]
pmatch varname, variables(<@\textcolor{OrangeRed}{varlist}@>) body(
    [pattern1 => exp1,]
    [pattern2 => exp2,]
)
\end{lstlisting}
\end{frame}

\begin{frame}{Patterns}
\begin{table}[]
\scriptsize
\renewcommand{\arraystretch}{1.5} 
\begin{tabular}{|l|c|l|}
\hline
 \footnotesize \bf{Pattern} & \footnotesize \bf{Syntax} & \footnotesize \bf{Description} \\ \hline
 Constant \; & $x$                                   & A simple value, a number, or a string. \\ \hline
 Range \;    & $a \sim b$                            & A range between $a$ and $b$. \\ \hline
 Or \;       & $pattern \; | \; ... \; | \; pattern$ & The union of multiple patterns for a variable. \\ \hline
 Wildcard \; & $\_$                                  & Any pattern that has not been matched yet. \\ \hline
 Tuple \;    & $(pattern, \; ..., \; pattern)$       & The intersection of patterns for different variables. \\ \hline
\end{tabular}
\end{table}
\end{frame}

\begin{frame}[fragile]{Example 1: Constant pattern}
\label{example_1}

\footnotesize
\begin{lstlisting}[language=Stata]
gen var_1 = ""
replace var_1 = "very low"  if rep78 == 1
replace var_1 = "low"       if rep78 == 2
replace var_1 = "mid"       if rep78 == 3
replace var_1 = "high"      if rep78 == 4
replace var_1 = "very high" if rep78 == 5
replace var_1 = "missing"   if rep78 == .
\end{lstlisting}

\begin{lstlisting}[language=Stata]
pmatch var_2, variables(rep78) body( ///
    1 => "very low",                 ///
    2 => "low",                      ///
    3 => "mid",                      ///
    4 => "high",                     ///
    5 => "very high",                ///
    . => "missing",                  ///
)
\end{lstlisting}

\tikz[remember picture, overlay] {\node[anchor=south east, outer sep=10pt] at (current page.south east) {\hyperlink{example_6}{\beamerbutton{Example 6}}};}
\end{frame}

\begin{frame}[fragile]{Example 2: Range pattern}
\label{example_2}

\footnotesize
\begin{lstlisting}[language=Stata]
gen var_1 = ""
replace var_1 = "cheap"     if price >= 0    & price <  6000
replace var_1 = "normal"    if price >= 6000 & price <  9000
replace var_1 = "expensive" if price >= 9000 & price <= 16000
replace var_1 = "missing"   if price == .
\end{lstlisting}

\begin{lstlisting}[language=Stata]
pmatch var_2, variables(price) body( ///
    min<@$\newsim$@>!6000   => "cheap",          ///
    6000<@$\newsim$@>!9000  => "normal",         ///
    9000<@$\newsim$@>max    => "expensive",      ///
    .           => "missing",        ///
)
\end{lstlisting}

\textit{Note:} the $!$ excludes the boundary.
$a \newsim b$ includes $a$ and $b$, $a \newsim ! b$ includes $a$ but not $b$, $a !\newsim b$ excludes $a$ and includes $b$. $a !! b$ excludes both $a$ and $b$.

\tikz[remember picture, overlay] {\node[anchor=south east, outer sep=10pt] at (current page.south east) {\hyperlink{example_7}{\beamerbutton{Example 7}}};}
\end{frame}

\begin{frame}[fragile]{Example 3: Or pattern}
\label{example_3}

\footnotesize
\begin{lstlisting}[language=Stata]
gen var_1 = ""
replace var_1 = "low"     if rep78 == 1 | rep78 == 2
replace var_1 = "mid"     if rep78 == 3
replace var_1 = "high"    if rep78 == 4 | rep78 == 5
replace var_1 = "missing" if rep78 == .
\end{lstlisting}

\begin{lstlisting}[language=Stata]
pmatch var_2, variables(rep78) body( ///
    1 | 2  => "low",                 ///
    3      => "mid",                 ///
    4 | 5  => "high",                ///
    .      => "missing",             ///
)
\end{lstlisting}
\end{frame}

\begin{frame}[fragile]{Example 4: Wildcard pattern}
\label{example_4}

\footnotesize
\begin{lstlisting}[language=Stata]
gen var_1 = "other"
replace var_1 = "very low" if rep78 == 1
replace var_1 = "low"      if rep78 == 2
\end{lstlisting}

\begin{lstlisting}[language=Stata]
pmatch var_2, variables(rep78) body( ///
    1 => "very low",                 ///
    2 => "low",                      ///
    _ => "other",                    ///
)
\end{lstlisting}
\end{frame}

\begin{frame}[fragile]{Example 5: Tuple pattern}
\label{example_5}

\footnotesize
\begin{lstlisting}[language=Stata]
gen var_1 = ""
replace var_1 = "case 1"  if rep78 <  3 & price <  10000
replace var_1 = "case 2"  if rep78 <  3 & price >= 10000
replace var_1 = "case 3"  if rep78 >= 3
replace var_1 = "missing" if rep78 == . | price == .
\end{lstlisting}

\begin{lstlisting}[language=Stata]
pmatch var_2, variables(rep78 price) body( ///
    (min<@\scriptsize{$\sim$!}@>3, min<@$\newsim$@>!10000) => "case 1",      ///
    (min<@$\newsim$@>!3, 10000<@$\newsim$@>max)  => "case 2",      ///
    (3<@$\newsim$@>max,  _)          => "case 3",      ///
    (., _) | (_, .)      => "missing",     ///
)
\end{lstlisting}
\end{frame}

\begin{frame}{Checks}
Convenient syntax, but that’s not the main benefit. \\~
\begin{itemize}
    \item Exhaustiveness
    \begin{itemize}
        \item Did you forgot some cases ? \\~
    \end{itemize}
    \item Usefulness
    \begin{itemize}
        \item Are all the conditions useful ?
        \item Are there some overlaps between them ? \\~
    \end{itemize}
\end{itemize}

No time for the algorithm, straight to the results.
\end{frame}

\begin{frame}[fragile]{Example 6: Exhaustiveness}
\label{example_6}

\footnotesize
\begin{lstlisting}[language=Stata]
gen var_1 = ""
replace var_1 = "very low"  if rep78 == 1
replace var_1 = "low"       if rep78 == 2
replace var_1 = "mid"       if rep78 == 3
replace var_1 = "high"      if rep78 == 4
replace var_1 = "very high" if rep78 == 5
\end{lstlisting}

\begin{lstlisting}[language=Stata]
pmatch var_2, variables(rep78) body( ///
    1 => "very low",                 ///
    2 => "low",                      ///
    3 => "mid",                      ///
    4 => "high",                     ///
    5 => "very high",                ///
)
    <@\textcolor{red}{Warning : Missing values}@>
        <@\textcolor{red}{.}@>
\end{lstlisting}

\tikz[remember picture, overlay] {\node[anchor=south east, outer sep=10pt] at (current page.south east) {\hyperlink{example_1}{\beamerbutton{Example 1}}};}
\end{frame}

\begin{frame}[fragile]{Example 7: Overlaps}
\label{example_7}

\footnotesize
\begin{lstlisting}[language=Stata]
gen var_1 = ""
replace var_1 = "cheap"     if price >= 0    & price <= 6000
replace var_1 = "normal"    if price >= 6000 & price <= 9000
replace var_1 = "expensive" if price >= 9000 & price <= 16000
replace var_1 = "missing"   if price == .
\end{lstlisting}

\begin{lstlisting}[language=Stata]
pmatch var_2, variables(price) body( ///
    min<@$\newsim$@>6000  => "cheap",            ///
    6000<@$\newsim$@>9000 => "normal",           ///
    9000<@$\newsim$@>max  => "expensive",        ///
    .         => "missing",          ///
)
    <@\textcolor{red}{Warning : Arm 2 has overlaps}@>
        <@\textcolor{red}{Arm 1: 6000}@>
    <@\textcolor{red}{Warning : Arm 3 has overlaps}@>
        <@\textcolor{red}{Arm 2: 9000}@>
\end{lstlisting}

\tikz[remember picture, overlay] {\node[anchor=south east, outer sep=10pt] at (current page.south east) {\hyperlink{example_2}{\beamerbutton{Example 2}}};}
\end{frame}

\begin{frame}[fragile]{Example 8: Usefulness}
\label{example_8}

\footnotesize
\begin{lstlisting}[language=Stata]
gen var_1 = ""
replace var_1 = "cheap"     if price >= 0    & price <  6000
replace var_1 = "normal"    if price >= 6000 & price <  9000
replace var_1 = "expensive" if price >= 9000 & price <= 16000
replace var_1 = "missing"   if price == .
\end{lstlisting}

\begin{lstlisting}[language=Stata]
pmatch var_2, variables(price) body( ///
    min<@$\newsim$@>!6000  => "cheap",           ///
    6000<@$\newsim$@>!9000 => "normal",          ///
    9000<@$\newsim$@>max   => "expensive",       ///
    min<@$\newsim$@>max    => "oops",            ///
    .          => "missing",         ///
)
    <@\textcolor{red}{Warning : Arm 4 is not useful}@>
    <@\textcolor{red}{Warning : Arm 4 has overlaps}@>
        <@\textcolor{red}{Arm 1: 3291$\newsim$5999}@>
        <@\textcolor{red}{Arm 2: 6000$\newsim$8999}@>
        <@\textcolor{red}{Arm 3: 9000$\newsim$15906}@>
\end{lstlisting}
\end{frame}

\begin{frame}{Limitations}
What does it cost compare to `replace … if …` statements ?
\begin{itemize}
    \item It depends on your data
    \item The command has 4 steps
    \begin{itemize}
        \item Checking the variables
        \item Parsing the body
        \item Checking the conditions
        \item Evaluating each arm
    \end{itemize}
    \item $< 1$M observations, it’s less than 0.1s
    \item $\geq 1$M observations, checking levels becomes costly
\end{itemize}

\end{frame}

\begin{frame}{Next steps}
\label{next_steps}

Supports byte, integer, long, float, double, and strings
\begin{itemize}
    \item Already supports using label values instead of values \hyperlink{example_9}{\beamerbutton{Example 9}}
    \item Plan to add support for dates
    \item Plan to add \textit{missing} and \textit{nonmissing} patterns
    \item Plan to add examples in the warnings
    \item Plan to add possibility to ignore impossible cases with tuples
\end{itemize}
\end{frame}

\begin{frame}[fragile]{Conclusion}
This project is still young, this is my first time presenting it
\begin{itemize}
    \item Tell me if you find it interesting, or what you think are the issues
    \item Comments on the syntax, features, or anything else are welcomed \\~
\end{itemize}

You can find the project and the installation command on GitHub
\begin{verbatim}
    https://github.com/MaelAstruc/stata_match
\end{verbatim}

You can contact me by email
\begin{verbatim}
    mael.astruc-le-souder@u-bordeaux.fr

\end{verbatim}

\begin{center}
Thank you for your attention!
\end{center}

\end{frame}

\begin{frame}[fragile, noframenumbering]{Example 9: Label values}
\label{example_9}

\footnotesize
\begin{lstlisting}[language=Stata]
drop _all
set obs 100
gen int color = runiform(1, 4)
label define color_label 1 "Red" 2 "Green" 3 "Blue"
label values color color_label 
\end{lstlisting}

\begin{lstlisting}[language=Stata]
pmatch color_hex, variables(color) body( ///
    1     => "#FF0000",                  ///
    2     => "#00FF00",                  ///
   "Blue" => "#0000FF",                  ///
)
\end{lstlisting}

\tikz[remember picture, overlay] {\node[anchor=south east, outer sep=10pt] at (current page.south east) {\hyperlink{next_steps}{\beamerbutton{Next steps}}};}
\end{frame}

\end{document}